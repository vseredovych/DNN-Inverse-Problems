% ============================================ %

\documentclass[14pt,a4paper]{extarticle}
%\documentclass[12pt,a4paper]{article}

\usepackage[utf8]{inputenc}
\usepackage[ukrainian]{babel}


\usepackage{amssymb}
\usepackage{physics}
\usepackage{amsmath}
% \operatorname*{argmin}_\theta f(x)
% \operatorname*{arg\,max}_\theta f(x)



\usepackage[active]{srcltx}
\usepackage[final]{pdfpages}

\usepackage[hidelinks]{hyperref}

\usepackage{verbatim}

% ============================================ %

%\pagestyle{empty}                     %нумерацiя сторiнок i т.д.
\pagestyle{headings}                   %нумерацiя сторiнок вгорi зправа i т.д.
%\renewcommand{\baselinestretch}{1.5}   %мiжстрiчковий інтервал
%\parindent=7.5mm                      %абзацний відступ
\righthyphenmin=2                     %перенос 2 останніх букв
\pagenumbering{arabic}
\tolerance=400
\mathsurround=2pt
\hfuzz=1.5pt

% ============================================ %

\hoffset=-0.5cm        %+2.5cm -- вiдступ вiд лiвого краю
\voffset=-1.5cm        %+2.5cm -- вiдступ зверху
\oddsidemargin=0.1cm   %ліве поле
\topmargin=0.1cm       %верхнє поле
\headheight=0.5cm      %висота верхнього колонтитулу
\footskip=1cm          %висота нижнього колонтитулу
\headsep=0.3cm         %відступ від колонт. до тексту
\textwidth=17cm        %ширина сторінки
\textheight=25.5cm     %висота сторінки

% ============================================ %
	
\newcounter{e}
\setcounter{e}{0}
\newcommand{\n}{\refstepcounter{e} (\arabic{e})}

\newcounter{pic}
\setcounter{pic}{0}
\newcommand{\pic}[1]{\refstepcounter{pic} \vspace{-0.3cm}\textit{Рисунок \arabic{pic}\label{#1}.}}

\newcounter{tabl}
\setcounter{tabl}{0}
\newcommand{\tabl}[1]{\refstepcounter{tabl} \vspace{-0.3cm}\textit{Таблиця \arabic{tabl}\label{#1}.}}

\newcounter{dod}
\setcounter{dod}{0}
\newcommand{\dod}[1]{\refstepcounter{dod} \textit{Додаток \arabic{dod}\label{#1}.}}


\newtheorem{theorem}{Теорема}[section]
\newtheorem{defn}[theorem]{Означення}
\newtheorem{lemma}[theorem]{Лема}

\newcommand{\proof}{\textit{Доведення. \space}}
% \setcounter{page}{1}
% \setcounter{section}{1}

\numberwithin{equation}{section}
\numberwithin{figure}{section}

% \newcommand{\unknownx}{	\boldsymbol{$1}^{\star}}
% \newcommand{\bt}[1]{\textbf{#1}}

% ============================================ %
	
% bibliography
\usepackage[
	backend=biber,
	style=numeric,
	sorting=none
]{biblatex}
\addbibresource{resources/bibliography.bibtex}

% ============================================ %

\begin{document}
	% ============================================ %
	\begin{titlepage}%
		\begin{center}
			{\textbf{ЛЬВІВСЬКИЙ НАЦІОНАЛЬНИЙ УНІВЕРСИТЕТ \\ ІМЕНІ ІВАНА ФРАНКА}}\par
			{Факультет прикладної математики та інформатики \\ Кафедра обчислювальної математики}\par
			\begin{center}
				
			\end{center}
			\vspace{25mm}
			{\textbf{\huge{Курсова робота}}}\par
			\vspace{5mm}
			{\large{Використання глибокого навчання для обернених задач}}\par
			\vspace{5mm}
			{}\par %subtitle
		\end{center}
		
		\vfill
		\vskip80pt
		
		\begin{flushleft}
			\hskip 8cm 
			Виконав студент IV курсу групи
			\\ \hskip8cm
			ПМп-41 напрямку підготовки 
			\\ \hskip8cm
			(спеціальності)
			\\ \hskip8cm
			113 -- ``Прикладна математика''
			\\ \hskip8cm
			Середович В.B.
		\end{flushleft}
		\begin{flushleft}
			\hskip8cm 
			Курівник: Музичук Ю.А
		\end{flushleft}
		
		\vfill
		
		\begin{center}
			\large
			Львів - 2020
		\end{center}
	\end{titlepage}

	% ============================================ %
	% Зміст
	\addtocontents{toc}{\protect\thispagestyle{empty}}
	\tableofcontents

	% ============================================ %
	% Вступ
	
	\newpage
	\thispagestyle{empty}
	\addcontentsline{toc}{section}{Вступ}
	\section*{Вступ}
	
	Вступ про типи обернені задач та глибоке навчання
	\\
	
	Оберненими задачами називають такі задачі, коли необхідно відновити параметри які характеризують деяку модель з використанням непрямих спостережень. До них можна віднести багото по відновленню зображень зменшення кількості шуму (deblurring) чи заповнення втрачених даних (inpainting).

	% ============================================ %
	
	\newpage
	\thispagestyle{empty}
	\section{Постановка задачі} 
	Оберненими задачами будемо вважати такі задачі, в яких невідомим є $n-$ піксельне зображення $\boldsymbol{x}^{\star} \in \mathbb{R}^{n}$ яке було отримане з $m$ вимірювань $\boldsymbol{y} \in \mathbb{R}^{m}$ з певним рівнем шуму $\boldsymbol{\varepsilon}$
	$$
	\boldsymbol{y}=\mathcal{A}\left(\boldsymbol{x}^{\star}\right)+\boldsymbol{\varepsilon}
	$$
	
	де $\mathcal{A}$ - це прямий оператор вимірювання та $\varepsilon$ є вектором шуму. Метою є відновлення $x^{\star}$ з $y$. Можна розглянути більш загальний випадок моделі неадитивної шуму, який має вигляд 
	$$
	\boldsymbol{y}=\mathcal{N}\left(\mathcal{A}\left(\boldsymbol{x}^{\star}\right)\right)
	$$
	де $\mathcal{N}(\cdot)$ є прикладами вибірки з шумом.

	Цей процес називають некоректним або погано поставленим (ill-posed), бо реконстроювати єдиний розв'язок який задовілняє спостереження є складною або неможливою задачей за умови відсутності попередніх знаннь про дані.

	Отже метою даної рототи буде розв'язання таких обернених задач з використанням моделей глибокого навчання.

	

	%	Метою даної роботи є дослідження ефективності різних методів атак на лінійні моделі машинного навчання, та аналіз можливих методів захисту від них.
		
	%	Виходячи з мети, визначеними завданнями роботи є:
	%	\begin{itemize}
	%		\item Реалізувати лінійну модель машинного навчання
	%		\item Розглянути різні методи генерування змагальних прикладів
	%		\item Застосувати атаки на створену модель та проаналізувати їх ефективність
	%		\item Розглянути можливі методи захисту від атак
	%	\end{itemize}
		
	% ============================================ %
	\newpage
	\thispagestyle{empty}
	\section{Методи розв'язання обернених задач}

	\subsection{Метод регуляризації}
	
	Якщо розподіл шуму відомий, розвязання задачі оцінки максимальної ймовірності (Maximum Likelihood), може відновити $x$
	$$
	\hat{\boldsymbol{x}}_{\mathrm{ML}}
	=\underset{\boldsymbol{x}}{\arg \max{ p (\boldsymbol{y} \mid \boldsymbol{x}) }}
	=\underset{\boldsymbol{x}}{\arg \min }-\log p(\boldsymbol{y} \mid \boldsymbol{x})
	$$
	де $p(\boldsymbol{y} \mid \boldsymbol{x})$ це ймовірність спостереження $\boldsymbol{y}$ за умови якщо $\boldsymbol{x}$ є справжнім зображенням.
	
	В залежності від умов задачі, можуть бути відомі попередні дані про те яким має бути $x$. Ці умови можуть бути використанні для формування  задачі оцінки максимальної апостеріорної ймовірності

	$$
	\hat{\boldsymbol{x}}_{\mathrm{MAP}}
	=\underset{\boldsymbol{x}}{\arg \max{ p(\boldsymbol{x} \mid \boldsymbol{y}) }}
	=\underset{\boldsymbol{x}}{\arg -\max{ p(\boldsymbol{y} \mid \boldsymbol{x})} } p(\boldsymbol{x})
	=\underset{\boldsymbol{x}}{\arg \min }-\ln p(\boldsymbol{y} \mid \boldsymbol{x})-\ln p(\boldsymbol{x})
	$$
	Для випадку білого гаусівського шуму (AVGN), з максимальної апостеріорної ймовірності виплаває:
	$$
	\underset{\boldsymbol{x}}{\arg \min } \frac{1}{2}\|\mathcal{A}(\boldsymbol{x})-\boldsymbol{y}\|_{2}^{2}+r(\boldsymbol{x})
	$$
	де  $r(\boldsymbol{x})$ є пропорційним до негативного логарифмічного пріора. $\boldsymbol{x} .$ 	Прикладами такого підходу є регуляризація Тіхонова.
	
	% total variation

	\subsection{Машинне навчання для обернених задач}
	TODO
	
	% ============================================ %
	\newpage
	\thispagestyle{empty}
	\section{Глибоке навчання для обернених задач}
	
	\subsection{Алгоритм 1}
	TODO

	\subsection{Алгоритм 2}
	TODO

	% ============================================ %
	\newpage
	\thispagestyle{empty}
	\section{Реалізація та аналіз}
	
	\subsection{Реалізація}
	TODO

	\subsection{Аналіз}
	TODO
	
	% ============================================ %		
	\newpage
	\thispagestyle{empty}
	\section{Висновок}
	TODO
	
	%============================================ %
	\newpage
	\thispagestyle{empty}
	
	% Deep Learning Techniques for Inverse Problems in Imaging
	\nocite{ongie2020deep}
		
	% Solving ill-posed inverse problems using iterative deep neural networks
	\nocite{Adler_2017}
	\printbibliography[title={Бібліографія}]
	% ============================================ %
\end{document}